\documentclass[a4paper, 12pt]{article}
\usepackage[top=1in, bottom=1in, left=1.25in, right=1.25in]{geometry}
\usepackage[utf8]{inputenc}
\usepackage{hyperref}
\usepackage{graphicx}
\usepackage{pifont} % star symbols
\renewcommand{\baselinestretch}{1.5} % line spacing
\usepackage{setspace}

\setcounter{secnumdepth}{0}

\begin{document}

% Personal Data
\begin{center}
  {\LARGE{IRINA IVANOVA}}\\
  \textit{experienced software tester and web enthusiast}
\end{center}
\href{https://it.irina-ivanova.eu}{it.irina-ivanova.eu}
\hspace{4.6cm}
irina.ivanova@protonmail.com\\
\href{https://github.com/iriiiina}{github.com/iriiiina}
\hspace{4.4cm}
+372 555 64154

% Work Experience
\section{Work Experience}

\textbf{Senior QA Specialist}, \href{https://nortal.com}{Nortal AS}, March 2010 \--- present\\
\indent | Electronic Identity Project, August 2017 \--- present\\
\indent \indent \indent back-end test automation: Java, Groovy, REST, Spock, Docker\\
\indent | Digital Healthcare Project, March 2010 \--- August 2017\\
\indent \indent \indent web software testing, client support, automation of daily tasks\\
\textbf{Volunteer IT Manager}, Artha Rahaakadeemia, Aug 2017 \--- May 2019\\
\indent \indent \indent front-end and back-end on WordPress\\
\textbf{Volunteer IT Manager}, \href{https://www.lastefond.ee}{TUH Children's Foundation}, Jan 2013 \--- July 2014\\
\indent \indent \indent web page management, creation of a new web page on Drupal

% Side Projects
\section{Side Projects}

\textbf{Personal Page And Blog}, \href{https://irina-ivanova.eu}{irina-ivanova.eu}\\
\indent | front-end and back-end: HTML5, CSS3, Jekyll\\
\textbf{Personal Portfolio Page}, \href{https://it.irina-ivanova.eu}{it.irina-ivanova.eu}\\
\indent | front-end and back-end: HTML5, CSS3, KISS\\
\textbf{Personal Blog About Making Books}, \href{https://book.irina-ivanova.eu}{book.irina-ivanova.eu}\\
\indent | front-end and back-end: HTML5, Sass, Jekyll\\
\textbf{Online Converter mm-to-pt}, \href{https://mm-to-pt.irina-ivanova.eu}{mm-to-pt.irina-ivanova.eu}\\
\indent | front-end and back-end: HTML5, CSS3, Vanilla JavaScript\\
\textbf{Browser Extensions}, \href{https://gitlab.com/irina-ivanova-extensions}{gitlab.com/irina-ivanova-extensions}\\
\indent | Chrome and FireFox extensions for optimizing daily tasks\\
\textbf{Scripts}, \href{https://github.com/iriiiina/scripts}{github.com/iriiiina/scripts}\\
\indent | Bash and Python scripts to automate routine tasks\\

\pagestyle{empty}

\newpage

% Education
\section{Education}

\textbf{Bachelor of Science in Engineering}, \href{https://www.ut.ee/en}{University of Tartu}, July 2016\\
\indent | Version Update Automation Using Scripting Language Bash

% Courses And Certificates
\section{Courses And Certificates}

\textbf{Secure Logging Course}, \href{https://clarifiedsecurity.com/secure-logging-training/}{Clarified Security}, March 2017\\
\textbf{Web Application Security Course}, \href{https://www.clarifiedsecurity.com/web-application-security-training/}{Clarified Security}, May 2016\\
\textbf{Rapid Software Testing Course}, \href{http://www.satisfice.com/info_rst.shtml}{Satisfice, Inc.}, September 2011\\
\textbf{ISEB ISTQB Foundation Certificate in Software Testing}, August 2010

% Conferences
\section{Conferences}

\textbf{\href{https://nordictestingdays.eu/2019-2/}{Nordic Testing Days 2019}}, Tallinn, Estonia, May 2019\\
\textbf{\href{https://2018.geekout.ee/}{GeekOut 2018}}, Tallinn, Estonia, June 2018\\
\textbf{\href{https://nordictestingdays.eu/2017-2/}{Nordic Testing Days 2017}}, Tallinn, Estonia, June 2017\\
\textbf{Let's Test 2015}, Stockholm, Sweden, May 2015\\
\textbf{\href{https://kristjanuba.wordpress.com/2014/09/27/pest5-testing-craft-social-or-technical/}{PEST 5}}, Tallinn, Estonia, March 2015\\
\textbf{\href{http://agile.ee/2014/agile-saturday-xi/}{Agile Saturday XI}}, Tallinn, Estonia, October 2014\\
\textbf{\href{https://nordictestingdays.eu/2014-2/}{Nordic Testing Days 2014}}, Tallinn, Estonia, June 2014\\
\textbf{\href{http://agile.ee/2014/agile-saturday-x/}{Agile Saturday X}}, Tallinn, Estonia, February 2014\\
\textbf{\href{http://agile.ee/agile-saturday/agile-saturday-ix/}{Agile Saturday IX}}, Tallinn, Estonia, September 2013

% Public Talks
\section{Public Talks}

\textbf{\href{https://ivanova-irina.blogspot.com/2018/07/nortal-summer-university-workshop.html}{Getting Started With Web Testing}}, Nortal Summer University, June 2018\\
\textbf{\href{https://ivanova-irina.blogspot.com/2017/06/nortal-summer-university-workshop-about.html}{What Tools to Use to Test Better?}}, Nortal Summer University, June 2017\\
\textbf{\href{https://ivanova-irina.blogspot.com/2015/07/presentation-in-devclub-about-biases-in.html}{Biases in Software Testing}}, Tallinn DevClub, July 2015\\
\textbf{\href{https://ivanova-irina.blogspot.com/2015/06/my-first-ever-workshop-at-nortal.html}{Bug Bash}}, Nortal TechDay, May 2015

\pagestyle{empty}

\newpage

\begin{center}
\LARGE{Helena Jeret-Mäe}
\end{center}

\begin{center}
\textit{Former head on testing at \href{https://nortal.com}{Nortal}, September 2017}
\end{center}

Irina is a professional software tester in the best sense of the word. She is continuously developing and broadening her knowledge in testing and technology which makes her an invaluable team member and leader in her own right. Irina is willing to take the time and make the effort to learn new skills, and pass this knowledge on in a team setting or by publishing a blog post. With her calm attitude and considerable technical savvy, she's the one to work with in building quality software.

\begin{center}
\------------------------------------------
\end{center}

\begin{center}
\LARGE{Jarno Raid}
\end{center}

\begin{center}
\textit{Former project manager at \href{https://nortal.com}{Nortal}, November 2017}
\end{center}

Irina is always up to practice new technologies and gadgets or create something on her own in order to improve results and make life easier. Curiosity going hand in hand with ultimate correctness and ability to dig into the depths most testers wouldn't dare to dream of makes her unique.

\newpage

\begin{center}
\LARGE{Küllike Saar}
\end{center}

\begin{center}
\textbf{English}\\
\textit{CEO in \href{https://www.lastefond.ee}{Tartu University Hospital Children's Foundation}, February 2017}
\end{center}

Irina was very responsible and agile volunteer. Under her guidance a lot of IT related issues have been solved. Additionally she had high motivation and was very creative and flexible in her actions. Irina is definitely valuable employee.

\begin{center}
\------------------------------------------
\end{center}

\begin{center}
\textbf{Estonian (Original)}\\
\textit{Tegevjuht \href{https://www.lastefond.ee}{Tartu Ülikooli Kliinikumi Lastefondis}, veebruar 2017}
\end{center}

Irina oli väga kohusetundlik ja kiire tegutsemisega vabatahtlik. Tema eestvedamisel said paljud IT-alased probleemkohad lahenduse. Lisaks oli tal kõrge motivatsioon ning oma tegemistes oli ta väga loov ning samas ka paindlik/muutustega kohanev. Irina on kindlasti väärtuslik töötaja.

\newpage

% Agnes Karlson
{\setstretch{1.0}

\begin{center}
\LARGE{Agnes Karlson}
\end{center}

\begin{center}
\textbf{English}\\
\textit{Former CEO in \href{https://www.lastefond.ee}{Tartu University Hospital Children's Foundation} and former CEO at Artha Rahaadakeemia Foundation, January 2017}
\end{center}

Irina Ivanova was a volunteer in Tartu University Hospital Children's Foundation from 2013 to 2014.\\

Irina helped with all day-to-day IT related issues in the organization, but her greatest project was to create platform and prototype for the new homepage according to the visuals provided by creative agency.\\ 

I am very thankful to Irina for all her time, that she was ready to contribute in the idea to evolve IT area of one of the largest children's oriented foundation in Estonia!\\

If all people would do voluntary work with such commitment as Irina did, then charitable organizations would evolve as common successful organizations!\\

\begin{center}
\------------------------------------------
\end{center}

\begin{center}
\textbf{Estonian (Original)}\\
\textit{Endine tegevjuht \href{https://www.lastefond.ee}{Tartu Ülikooli Kliinikumi Lastefondis} ja endine tegevjuht Artha Rahaakadeemia fondis, jaanuar 2017}
\end{center}

Irina Ivanova oli vabatahtlik 2013-2014 Tartu Ülikooli Kliinikumi Lastefondis, sel perioodil olin seal tegevjuht.\\

Irina aitas kõigi IT alaste jooksvate küsimustega fondis, aga suurimaks projektiks oli uue kodulehe platvormi ja prototüübi loomine 2013. aastal vastavalt loovagentuuri visuaalidele.\\

Ma olen väga tänulik Irinale, et ta oli nõus nii palju oma aega panustama sellele, et arendada ühe Eesti mõjukama lastele suunatud heategevusorganisatsiooni IT ala!\\

Kui kõik inimesed teeksid vabatahtlikku tööd sama suure pühendumisega, nagu Irina seda tegi, siis töötaksid heategevusorganisatsioonid nagu täiesti tavalised edukad ettevõtted!\\

\end{document}